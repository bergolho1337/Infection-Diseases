\documentclass[12pt]{article}

\usepackage{sbc-template}
\usepackage{graphicx,url}
\usepackage[utf8]{inputenc}
\usepackage{amsmath}
\usepackage{subfig}
\usepackage{float}
\usepackage[]{algorithm2e}
%\usepackage[brazil]{babel}
%\usepackage[latin1]{inputenc}  

     
\sloppy

\title{SIR Model}

\author{Lucas Arantes Berg\inst{1}}


\address{Modelagem Computacional -- Universidade Federal de Juiz de Fora
  (UFJF)\\
  Caixa Postal 20.010-- 36.016-970 -- Juiz de Fora -- MG -- Brazil
}

\begin{document} 

\maketitle

\begin{abstract}
   
\end{abstract}
     
\begin{resumo} 
  
\end{resumo}


\section{Introduction}

A brief introduction ...

\subsection{SIR Model}

The first well-know mathematical model for modelling diseases was the SIR model, which was first used by Kermack and McKendrick in 1927 and has subsequently been applied to a variety of diseases, especially airborne childhood diseases with lifelong immunity upon recovery, such as measles, mumps, rubella, and pertussis. 

The model is given by a set of ordinary differential equations (ODE's) and can supply two types of behaviours, with and without vital dynamics.

\begin{equation}
	\begin{cases}
		\frac{dS}{dt} = \mu N - \frac{\beta S I}{N} - \eta S \\
		\frac{dI}{dt} = \frac{\beta S I}{N} - \gamma I - \eta I \\
		\frac{dR}{dt} = \gamma I - \eta R,
	\end{cases}
\end{equation}
where $N = S + I + R = 1$ is the total population.

\begin{itemize}
	\item $S = $ Percentage of susceptible
	\item $I = $ Percentage of infectious
	\item $R = $ Percentage of recovered
	\item $\beta = $ Infection rate
	\item $\gamma = $ Recovery rate
	\item $\mu = $ Birth rate
	\item $\eta = $ Death rate
\end{itemize}


\bibliographystyle{sbc}
\bibliography{sbc-template}

\end{document}
